\documentclass{resume}

\name{Liam Rosenfeld}
\contact{
    \href{mailto:me@liamrosenfeld.com}{me@liamrosenfeld.com} \\ 
    \href{https://liamr.dev}{liamr.dev} \\
    407-864-0452
}

\newif\ifisDigital
\isDigitaltrue
% \isDigitalfalse

\begin{document}

%----------------------------------------------------------------------------------------
% Skills	
%----------------------------------------------------------------------------------------
\begin{rSection}{Skills}
\begin{tabular}{ @{} >{\bfseries}l @{\hspace{6ex}} l }
Languages & Swift, Rust, C++, Python, Javascript/Typescript, Java, SQL, ARM, VHDL \\
Frameworks & SwiftUI, AppKit, UIKit, Accelerate, Axum, Vue, Svelte \\
Tools & macOS, Linux, Git, Xcode Instruments, Docker, MongoDB, \LaTeX
\end{tabular}
\end{rSection}

%----------------------------------------------------------------------------------------
% Education
%----------------------------------------------------------------------------------------

\begin{rSection}{Education}

{\bf BS in Computer Science}, University of Florida \hfill {Class of 2025}\\
GPA: 4.0, Minor in Mathematics, Honors Program Member
\end{rSection}

%---------------------------------------------------------------------------------------
% Work History
%----------------------------------------------------------------------------------------
\begin{rSection}{Work History}
\vspace{-1.25em}

\item \textbf{Backend Software Engineer} \hfill Summer 2022 \\
Parametric Capital
\begin{itemize}
    \itemsep -3pt {} 
    \vspace{-0.35em}
    \item Built a server to collect, aggregate, and serve time series metrics to a visualization frontend
    \item Used Rust (with Axum and Tonic frameworks) for performance and MongoDB for storing structured data
    \item Designed and implemented a RESTful OpenAPI specification and a GRPC Protobuf specification
    % \item Encapsulated business goals as a RESTful OpenAPI-compliant API specification
    % \item Worked on a fast pace team and shipped version one while focusing on a maintainable architecture along with unit and integration tests
\end{itemize}

\end{rSection}

%---------------------------------------------------------------------------------------
% Projects
%----------------------------------------------------------------------------------------
\begin{rSection}{Select Projects}

\ifisDigital
\vspace{-1.25em}
\else
Writeups on my website
\fi

\item \textbf{Raspberry Pi Rust OS} \ {Built 2022} 
\ifisDigital\hfill\href{https://github.com/liamrosenfeld/rustos}{Source}\fi
\begin{itemize}
    \itemsep -3pt {}
    \vspace{-0.35em}
    \item A kernel and basic operating system for a Raspberry Pi built in Rust
    \item Implemented booting, GPIO, UART, chainloading, allocation, and a Fat32 filesystem
    \item Debugged using a JTAG and GDB
\end{itemize}

\item \textbf{Iconology} \ {Released 2020} 
\ifisDigital\hfill\href{https://liamrosenfeld.com/projects/iconology}{Writeup}\fi
\begin{itemize}
    \itemsep -3pt {}
    \vspace{-0.35em}
    \item macOS app to stream-line the process of icon generation with 5k downloads
    \item Built using AppKit, CoreGraphics, and SwiftUI
\end{itemize}

\item \textbf{Image to ASCII Art} \ {Released 2017}
\ifisDigital\hfill \href{https://liamrosenfeld.com/projects/ascii-art/}{Writeup}\fi
\begin{itemize}
    \itemsep -3pt {} 
    \vspace{-0.35em}
    \item iOS and macOS app on the App Store with 10k downloads
    \item Interface built in SwiftUI, UIKit, and AppKit
    \item Generator uses Accelerate vImage matrix operations
\end{itemize}

\item \textbf{Linkr} \ {Released 2020}
\ifisDigital\hfill \href{https://liamrosenfeld.com/projects/linkr/}{Writeup}\fi
\begin{itemize}
    \itemsep -3pt {} 
    \vspace{-0.35em}
    \item Self hosted URL shortener for organizations
    \item Built using Rust, Svelte, and PostgreSQL for long term stability
    \item Implemented at Full Sail University to automate changes to their learning management system
\end{itemize}

% \item \textbf{Open House} {Released 2020}
% \ifisDigital\hfill \href{https://liamrosenfeld.com/projects/open-house/}{Writeup}\fi
% \begin{itemize}
%     \itemsep -3pt {}
%     \vspace{-0.35em}
%     \item A website to manage volunteer organizations
%     \item Built with Vue in Typescript and uses Firebase for the backend
% \end{itemize}

% \item \textbf{Cloud Signage} {Released 2021}
% \begin{itemize}
%     \itemsep -3pt {} 
%     \item A digital signage solution for organizations
%     \item Includes both a macOS Screensaver and Standalone app
%     \item Integrates with Amazon S3
%     \item Implemented at Full Sail University for their computer labs and lobby signage
% \end{itemize}

\item \textbf{WWDC Scholar} \ {2019, 2020}
\ifisDigital
\hfill
\href{https://liamrosenfeld.com/projects/fourier-artist/}{2019 Writeup,}
\href{https://liamrosenfeld.com/projects/stfourier-explainer/}{2020 Writeup}
\fi
\begin{itemize}
    \itemsep -3pt {} 
    \vspace{-0.35em}
    \item My 2019 submission visualized the Fourier transform as rotating circles drawing a path
    \item My 2020 submission taught applying the Fourier transform to digital signal processing using Accelerate vDSP
    \item I had an opportunity to discuss my projects with Tim Cook
\end{itemize}
 
\end{rSection} 

%----------------------------------------------------------------------------------------
% Extra
%----------------------------------------------------------------------------------------
\begin{rSection}{Research}
\vspace{-1.25em}

\item \textbf{Lilypad} {2021-Present}
\begin{itemize}
    \itemsep -3pt {} 
    \vspace{-0.35em}
    \item Building a dual modal code editor to improve introductory programming education
    \item Working with Dr. Jeremiah Blanchard and seven undergrad and graduate students
    \item Building using Rust to run in the browser using Web Assembly
    \item Focused on the human computer interaction of reading and editing code
\end{itemize}

\end{rSection} 

\begin{rSection}{Teaching}
\vspace{-1.25em}

\item \textbf{Advanced Programming Fundamentals (COP 3504C) TA} \hfill {Fall 2022}
\item \textbf{AP Computer Science Principles TA} \hfill {2019-2020}

\end{rSection} 

\end{document}
